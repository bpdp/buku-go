\chapter{Pengenalan Go}

\section{Apa itu Go?}

Go adalah nama bahasa pemrograman sekaligus nama implementasi dalam bentuk kompilator (\textit{compiler}). Untuk pembahasan berikutnya, istilah ``Go'' akan mengacu pada spesifikasi bahasa pemrograman serta peranti pengembangannya.

\section{Lisensi Go}

Go didistribusikan dengan menggunakan lisensi modifikasi dari BSD. Lisensi lengkap dari Go bisa diakses di \url{http://golang.org/LICENSE}. Secara umum, penggunaaan lisensi ini mempunyai implikasi sebagai berikut:
\begin{itemize}
\item boleh digunakan untuk keperluan komersial maupun non-komersial tanpa batasan
\item boleh memodifikasi sesuai keperluan
\item boleh mendistribusikan
\item boleh memberikan sublisensi ke pihak lain
\item boleh memberikan garansi
\item tidak boleh menggunakan merk dagang Go
\item tanpa jaminan dan jika terjadi kerusakan terkait penggunaan software ini maka pemberi lisensi tidak bisa dituntut
\item jika mendistribusikan harus mengikutsertakan pemberitahuan hak cipta.
\end{itemize}

\lstset{language=Go, caption=hello.go}
\lstinputlisting{src/go/bab-01/hello.go}
