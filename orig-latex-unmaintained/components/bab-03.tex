\chapter{Dasar-dasar Pemrograman Go}

\section{Struktur Program Go}

\subsection{Program Aplikasi Sederhana - 1 File \textit{Executable} Utama}

Suatu aplikasi \textit{executable} (artinya bisa dijalankan secara langsung oleh sistem operasi) mempunyai struktur seperti yang terlihat pada Listing~\ref{lst:aplikasiGo}.

\lstset{language=Go, caption=hello.go, label={lst:aplikasiGo}}
\lstinputlisting{src/go/bab-03/aplikasi.go}

\subsection{Pustaka}


\section{Tipe Data Dasar}

\subsection{Angka / Numerik}

int  int8  int16  int32  int64
uint uint8 uint16 uint32 uint64 uintptr

byte // alias for uint8

rune // alias for int32
     // represents a Unicode code point

float32 float64

complex64 complex128


\subsection{string}

\subsection{bool}


\section{Variabel dan Konstanta}

\subsection{Variabel}

\subsection{Konstanta}

Konstanta dimaksudkan untuk menampung data yang tidak akan berubah-ubah. Konstanta dideklarasikan menggunakan kata kunci \textbf{const}. Konstant bisa bertipe \textit{character}, string, boolean, atau numerik.

\section{Pointer}

\section{Struktur Kendali}

\subsection{Perulangan dengan \textit{for}}


\subsection{Seleksi Kondisi}

\subsubsection{Pernyataan \textit{if}}


\subsubsection{Pernyataan \textit{switch}}


