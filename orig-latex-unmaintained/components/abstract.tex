\begin{abstract}
\thispagestyle{plain}
\setcounter{page}{1}
\addcontentsline{toc}{chapter}{\numberline{}Kata Pengantar}

Buku bebas ini merupakan buku yang dirancang untuk keperluan memberikan pengetahuan mendasar pengembangan aplikasi menggunakan Go (\url{http://golang.org}). Untuk mengikuti materi yang ada pada buku ini, pembaca diharapkan menyiapkan peranti komputer dengan beberapa software berikut terpasang:
\begin{itemize}
	\item Sistem operasi Linux (distribusi apa saja) - lihat di \url{http://www.distrowatch.com}. Semua software yang digunakan di buku ini sebenarnya juga bisa dijalankan pada sistem operasi lain (Windows dan MacOSX), tetapi jika ingin menggunakan selain Linux, silahkan membuat penyesuaian-penyesuaian sendiri.
	\item Compiler Go, bisa diperoleh di \url{http://golang.org}
	\item Git (untuk \textit{version control system}) - bisa diperoleh di \url{http://git-scm.com}
	\item mongoDB (basis data NOSQL) - bisa diperoleh di \url{http://www.mongodb.org}
	\item Vim (untuk mengedit source code) - bisa diperoleh di \url{http://www.vim.org}. Jika tidak terbiasa menggunakan Vim, bisa menggunakan editor teks lainnya (atau IDE), misalnya gedit (ada di GNOME), geany (\url{http://geany.org}), KATE (ada di KDE), dan lain-lain.
	\item Opsional: buku ini juga membahas penggunaan LiteIDE, software bebas yang berfungsi sebagai IDE dari Go. Software ini bisa diperoleh di \url{http://code.google.com/p/golangide/}.
\end{itemize}

Materi akan lebih banyak berorientasi ke command line / shell sehingga para pembaca sebaiknya sudah memahami cara-cara menggunakan shell di Linux.

Have fun!
\end{abstract}
